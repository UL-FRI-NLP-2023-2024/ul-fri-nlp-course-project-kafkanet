%%%%%%%%%%%%%%%%%%%%%%%%%%%%%%%%%%%%%%%%%
% FRI Data Science_report LaTeX Template
% Version 1.0 (28/1/2020)
% 
% Jure Demšar (jure.demsar@fri.uni-lj.si)
%
% Based on MicromouseSymp article template by:
% Mathias Legrand (legrand.mathias@gmail.com) 
% With extensive modifications by:
% Antonio Valente (antonio.luis.valente@gmail.com)
%
% License:
% CC BY-NC-SA 3.0 (http://creativecommons.org/licenses/by-nc-sa/3.0/)
%
%%%%%%%%%%%%%%%%%%%%%%%%%%%%%%%%%%%%%%%%%


%----------------------------------------------------------------------------------------
%	PACKAGES AND OTHER DOCUMENT CONFIGURATIONS
%----------------------------------------------------------------------------------------
\documentclass[fleqn,moreauthors,10pt]{ds_report}
\usepackage[english]{babel}

\graphicspath{{fig/}}

%----------------------------------------------------------------------------------------
%	ARTICLE INFORMATION
%----------------------------------------------------------------------------------------

% Header
\JournalInfo{FRI Natural language processing course 2023}

% Interim or final report
\Archive{Project report} 
%\Archive{Final report} 

% Article title
\PaperTitle{%
Conversations with Characters in Stories for Literacy
} 

% Authors (student competitors) and their info
\Authors{Blaž Erzar, Luka Salvatore Pecoraro, and Jakob Adam Šircelj}

% Advisors
\affiliation{\textit{Advisors: Slavko Žitnik}}

% Keywords
\Keywords{persona bot, role-playing, literature, education}
\newcommand{\keywordname}{Keywords}

%----------------------------------------------------------------------------------------
%	ABSTRACT
%----------------------------------------------------------------------------------------

\Abstract{%
In this project, we present the motivation for building models for
conversations with literary characters. By combining the existing
persona bot approaches with large language models, we propose solutions
to encourage curiosity and improve question-asking in pupils. We present
the existing literature explaining the state of reading curricula and
LLMs, enabling role-playing conversation.
}

%----------------------------------------------------------------------------------------

\begin{document}

% Makes all text pages the same height
\flushbottom 

% Print the title and abstract box
\maketitle 
 
% Removes page numbering from the first page
\thispagestyle{empty}

%----------------------------------------------------------------------------------------
%	ARTICLE CONTENTS
%----------------------------------------------------------------------------------------

\section*{Introduction}

Studies suggest that we are dealing with a literacy crisis \cite{nielen_digital_2018}. It is especially prevalent in pre-teens.
Curiosity is the fundamental driver of learning. One of the best ways to learn is to locate one's knowledge gaps and ask questions, which hopefully lead to answers that fill those gaps.

It turns out there are different tiers of questions and the ability to
formulate meaningful questions is both a rare trait in kids and a skill,
which can be improved. There are \textbf{surface-level} questions, e.g., 
\emph{Who was the main character?}, \textbf{conver\-gent-thinking}
questions, e.g., \emph{Why was the main character doing that in the
beginning?}, and \textbf{divergent-thinking} questions, e.g., \emph{What
would have happened if something else happened before ending instead?}.
The latter is the best in stimulating critical thinking because the
answers for them cannot be explicitly found in the text. They were also
very scarce in 5th graders, as found out in a study by Alaimi
\emph{et al.} \cite{alaimi_pedagogical_2020}. Studies have also shown that
interactive learning by asking divergent questions leads to a $20\%$ 
increase in the exams and make the absorbed knowledge more permanent
than linear learning in traditional educational systems. Potential reasons 
for that might be the inability to identify one's knowledge gaps,
fear of shame from asking a stupid question or suboptimal learning
environment.

A way of tackling this problem was proposed in \cite{alaimi_pedagogical_2020}. The idea was to create personaBots: LLM-based agents, which would interact with kids. After the kids were done reading some literary work, they would be able to ask their character of choice questions. This would, hopefully, stimulate their curiosity, to improve their question-asking ability, learning rate and critical thinking. They would be fine-tuned to their respective character counterparts from that particular literary work. 

\section*{Related work}

Some studies already suggest that worse literacy might do with the
quality of reading comprehension curriculums, as shown in
\cite{bogaerds-hazenberg_what_2022}. A better approach may be guided 
reading with digital pedagogical agents embedded in digital books,
as proposed in \cite{nielen_digital_2018}. This approach is similar
to using LLMs.

There already exist educational tools based on AI, which assist teachers
in creating lessons, e.g., Khanmigo. Based on prompts it suggests lecture
topics, plans, or even tutors pupils on solving them. But, it does not
focus on literacy. On the other hand, character.ai is a tool for creating
LLMs, whose answers resemble those of fictional, historic or other
characters. We can chat with these models, but they are not pedagogical
tools. For use with children, not all words should be permitted and
they should encourage curiosity and asking interesting questions. Both
of these tools are closed-source, so we do not know exactly how they
work.

Our focus will be on the work done in the field of role-playing, and personality modelling using LLMs. The concept of using LLMs for role-playing is described in detail by Shanan et. al. in \cite{shanahan_role-play_2023}. 
In the last few years, there were also many attempts to customize large language models, to role-play as fictional characters \cite{li_chatharuhi_2023, wang_incharacter_2024, shao_character-llm_2023, wang_rolellm_2023, chen_large_2023}. One of the most notable ones is ChatHaruhi \cite{li_chatharuhi_2023}, where researchers proposed a new approach for modelling fictional characters from Chinese and English literary, TV and anime characters. Authors Wang and others proposed RoleLLM, a framework to benchmark, elicit,
and enhance role-playing abilities in LLMs, along with providing large datasets for \cite{wang_rolellm_2023}.

By combining the role-playing capabilities of personaBots from character.ai, and the pedagogical focus of Khanmigo, we aim to build engaging personaBots based on fictional characters from literary works. The fictional personality will make our models more interesting to talk to than Khamingo, along with being more aligned for pedagogical purposes by encouraging question-asking and curiosity in users and having certain safeguards in place to prevent inappropriate responses.
Our approach to build such models will be based on the work described in the papers mentioned above.
We will use an existing LLM, and fine-tune it on a dataset of character
conversations, that we will build by extracting dialogues from novels,
plays and movie adaptations, along with using other LLMs to genereate new ones.
%----------------------------------------------------------------------------------------
%	REFERENCE LIST
%----------------------------------------------------------------------------------------
\bibliographystyle{unsrt}
\bibliography{report}

\end{document}
